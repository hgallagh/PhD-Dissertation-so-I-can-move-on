\chapter{Chapter 17}
\section{Photon Propagation: Modeling the Effects of Light Attenuation on Perception and Cognition}
\subsection{Well going into it}
So what is a fancy title for my light propagation model and how light travels in a vary rooms and lighting sources?  \\ 
\section{1.9.2022}
I honestly have less interest of being with people lately. I'm not playing games. I mean I basically claimed "mental health issues" when I wrote a dissertation and book since I didn't like anyone at the school I was at and know that I'm serving my karma there.
\\
Looking for remote jobs since I've always been more technical anyways it's really your confidence and how much you know that you can work for other people and not just for yourself and still have skills that you can be consulted for. 
\\ 
I want funding to attend conferences. I need grants. \\
I think I deserve to work at somewhere like google or harvard, but at the very least I can act like what I think someone would do. 
\\ 
In other news I'm happy when I'm not near my parents. I do not need that guilt complex of what I'm not doing I can help them when I'm available. 



\title{Interview Question if someone asked me about math modeling and at the very least a skeleton for review for this course and maybe why I failed it so many times because I kept on wanting to do this complicated well written assignment with graphs, figures, and citations like how I used to edit papers but the moment it was questioned I stopped }
\author{Hannah Gallagher}
\date{December 2021}


\maketitle

\section{In this current moment my work is embarrassing for my skill level and iq, think about the grit book and just type, subsequent damage control of losing cool/ professionalism next week}

\section{I'm actually really good at math. You're kind of bad at math.}

One week of being your idea of a bitch because frankly all of this emotional load was put on me as if my work is not important since it's technically school. No one cared about stressing me out and my visual memory for math doesn't matter to anyone. Oh I'm busy Hannah actually no fuck you. People only pay attention when you lose your shit otherwise they'll walk all over you. Or even better I just exist so I'm everyone's idea of a romantic interest that I had to stop wearing makeup here. What was the worst thing about learning math/optics? The men in the degree that thought it was okay to hit on me. No whining to people that annoy you. Academic advisors are not the people to whine to you since they aren't on your side. Give a piece of your mind to someone else where it won't come back to bite you.  

\section{When you know your worth now}
I remember when I lost my mind. 
\section{So you're basically giving yourself at the least the weekend to work on this for your own self esteem then you can go back to research responsibilities...probably would have been good to do this several months ago...but I didn't so I'll do it now. The pressure is off and everyone is watching tv to numb reality these days that I might as well just write my projects. Everything you did was really unhealthy for people you don't actually care about.}

You know this stuff why are you doing this to yourself. 

If you say things like you understand math modeling enough to not have to take it again maybe do something to back it up. It's your assignment and work at the end of the day so idk. I don't really want to do anything else until this is done and fleshed out the way I wanted it. 

\section{Elevator Problem}

\subsection{Math Required}







\subsection{From My Courses}

\subsection{Novice Level}

Upload the problem sets in a moment then figures via google drive. 
I used statistics and sets to figure out that it's best to have 2 and 4 floor people go on it. 


\subsection{Amateur Level}
I used random variables and pdfs. 


\subsection{Undergrad Level}

\subsection{Something that would demonstrate a good analysis}








\subsection{The Problem and Assignment in question}
It was in the form of memos to see how you could formulate a problem. It was trying to get you to imagine a situation when you are working to take this into question. \\ 

Don't even think about the calculations yet because look partial fraction calculations and derivations are simply that and I've already proved to myself I know how to do that. The exercise does have merit don't get me wrong and it's just another compartment for more. \\


So there is this idea of how the hypergeometric distribution could affect things. Then there are the matter of 

        
Hypergeometric is a thing then you could take the different probabilities for that.  

Then you could take the linear difference equations. Difference equations of logistic growth models. 




\subsection{What I would tell people}

\subsection{Novice}


Well at the end of the day this is a common interview question for coders to test if they can break down the ideas into manageable chunks and not panic. \\

Any math I can just write out on a piece of paper or the whiteboard and or say what I would do to plot a function. 

The simplest idea would be how this set of people are walking into a building. In this case we could use random variables. 

In this case we were using basic statistics to evaluate the situation. It used a basic pdf from introductory statistics. \\ 
%It was super annoying so I didn't do it. \\






\section{BloodCo2}

Well this would test how one could simulate and analyze the basal metabolic rate of a person. Think of this as a clinical trial. These people are coming into the hospital with varying health conditions and the people at the doctors at the hospital are asking you to analyze the situation for them. \\ 

This assignment was on linear difference equations. \\ 

By finding the root of the polynomials and using basic separation of variable techniques. One would arrive at the conclusion of where these roots would be. \\ 

This assignment also emphasized dynamical systems and where the steady state would be found. \\

A subsequent graph that converges to equilibrium would be useful to demonstrate that. 


\section{Fish Harvesting Logistic Growth Model}
Where you would see this in practice. A fish farm or a fish ecosystem such as the great lakes. \\  
\subsection{}
This is where the ode45 problem was introduced. 
I could assume that the fish are growing at the standard $e^-kt$ model.  

\subsection{What this assignment was about}

It was





\section{Predator Prey Model}

This is common in the field of epidemiology and gained more interest in the past year or two. 

The assumptions would be how these lions are 


\subsection{What you would tell people come on you know this just need a nudge...this is for you to look back on at this point...have a skeleton...it would be better to have these assignments typed up for reference so you don't need to think about it ever again}


\subsection{Plots on convergence}




\subsection{What was this assignment about}
The lotka voltera model and those subsequent equations. SIR Models. 

\subsection{What was the end product}




\section{Gravity Problem}


\subsection{Novice Level}
I used dimensional analysis to think about how different size animals could fall from this building by assuming as a sphere or disk for air resistance as the joke goes. 


\subsection{What was this assignment about}
Well there's these animals being dropped off a tall building of varying size. Think of the gravity experiment Galileo did of dropping different objects from the leaning tower of Pisa. \\ 

\subsection{What was the desired end product}
A demonstration of dimensional analysis. Thinking about how these models of gravity could grow in complexity. 

\subsection{Notes on it}




\section{Light Thesis}


\subsection{Novice Level}
I used differential equations and basic population models to think about how light was bouncing back and forth. 

My initial idea was to use partial differential equations and compartment models to optimize the function of a human eye which at the end of the day is a lens system. \\ 

There would be a few different compartments in question: \\ 
The light source \\

The medium it was traveling in and distance \\ 

the object being perceived \\ 

\subsection{What was the desired outcome}
An analysis of 

