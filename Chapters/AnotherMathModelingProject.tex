\documentclass{article}
\usepackage[utf8]{inputenc}

\title{The mess that is light}
\author{Hannah Gallagher}
\date{October 2019}

\begin{document}

\maketitle

\section{Introduction}
There's these things called light particles and we do things with them the end. For this project I have been going on long walks looking at water waves and the current brightness of the environment I'm in (indoors, outdoors, cloudy, dusk, warm light, blue light etc.) There's apps that can track lumens and vector works for lighting patterns so possible something with a grid of where the light is and it's subsequent intensity. 

\par 
The pop science of looking at waves right now is everything with graviational waves and how they behave. It's one of those ideas that these ideas may not seem important, but understanding the fundamental laws that govern our existence has led to sophisticated inventions. There's a great deal with fiber optic communication that you can use to understand how these waves diminish. I mean think about ultraviolet versus infrared or microwave versus gamma. Those all produce very different effects on people and the subject matter being exposed to this or that. The idea behind looking at water waves is that you're building an intuition and a visual representation for the things you cannot see such as photons. 

\section{A presentation of the problem in the application field.}

There are problems of bright light and how light affects are sleep and overall life patterns. 

\section{A discussion of the quantitative elements of the application field problem that are to be addressed by the mathematical model.}
There are these things with the flow of this light and water depending on if there are barriers in between them. 

\section{Data if you have some.} 
So much data of the photos I have been taking and the software in vector works I've been playing around with. 

\section{An account of the assumptions and approximations that the model involves.}

I'm assuming that light and water behave in the same way and that the medium in which they flow is not affected by other things such as humidity or the minerals in the affected water. 


\section{A thorough account of the variables and parameters of the model and their units.}

There are some fun greek letters that I will pull out of a hat and say that this $\gamma$ is now my variable for light. 

\section{A derivation of the mathematical model that includes an explicit account of the meaning and justification of every term in every equation.} 
I will derive this great movement of light and interference by the idea of probability and it must always add up to one and we only have these points. 


\section{A clear mathematical statement of the model; generally, this will be a system of equations, but it may involve inequalities or other elements.} 

This model models the light moving around and doing things. 

\section{An argument that the mathematical statement constitutes a well-posed problem. This needn’t be a proof, though it may be. The argument may rely on established mathematical theory, e.g., if your model is expressed as a system of ODEs, you may cite the relevant existence and uniqueness theorems from your favorite textbook; if your model is expressed as an eigenvalue problem for a matrix, you may cite the spectral theory of linear algebra.}

Well my favorite textbook authors include, but are not limited to: Goode, Zill, Walker, Nagle, Caroll, Burden,   
\section{A discussion of any quirky elements of the equations.}

This elements are so unique as they come from the thin air of people and light interacting with people and it only works once it exceeds this barrier otherwise it is still convergent, but it still diverges. 

\section{A discussion of the qualitative behavior of solutions. You should provide arguments to support your claims here. You may provide proofs, but you needn’t. You may provide citations to standard texts and to sources in the literature.} 

Good good literature from Caroll. 

\section{Particular solutions, if they bear on the application field problem or on the structure of the mathematical problem.} 
Have this much light exposure and this level of brightness for meaningful and productive work. 

\section{A discussion of how the equations are to be used to help solve the application field problem.}
Light and the idea of finding when the world is exposed to so much light when should people actually be going to bed? 
\section{The closed-form solution of the problem, if you have it.} 

Light is cool and the equations helped by solidifying the convergence. 

\section{A discussion of the numerical solution of the problem, if numerical solution is required. You needn’t lay out the numerical analysis in detail, but you should cite appropriate textbooks or literature and you should identify and discuss any numerical subtleties or complexities.} 

Great numerical solutions. 

\section{Examples of numerical solutions or the numerical solution if there is one. You should explain how you did the calculations; it’s fine to use packaged software, but do explain how you used it.} 
Numerical solutions of R doing things that I know how to do since I know how to type and program in these markup languages because I'm so great. 

\section{Graphs that illuminate the material.}
I will have these great graphs in maple that will illuminate the material and with vectorworks that do these things and it will look so sleek and professional. 

\section{A conclusion that summarizes the application field problem and the model and in which you explain how the model contributes to the solution of the application field problem} 

We found some great applications and further improvements of how we use and distribute light. 


\bibliography{}

\end{document}
